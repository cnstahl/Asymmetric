\section{Introduction} \label{sec:intro}


There are two robust, long-term behaviors that isolated quantum systems can demonstrate, thermalization and localization~\cite{Nandkishore2015}. In a localized system, initially local perturbations can be robust to the dynamics of the system and not diffuse away~\cite{Anderson58}. Thermalizing systems, on the other hand, equilibrate to a thermal state, which can be described by a small number of parameters, such as energy, entropy, pressure, etc. This equilibration occurs in the thermodynamic limit of large size and long time. 

A seeming contradiction is that quantum systems evolve unitarily, so initial information can not be lost (in a closed system). This also means the long-term behavior can not be fully described by the previously mentioned small number of parameters, so the full system will not thermalize. Connecting the system to a bath seems to solve this problem, but really we have only extended the problem to the larger Hilbert space that contains the system and bath. 

The resolution is that the initial information in a subsystem spreads throughout the system. The full information about the initial conditions is present within the entire system at later times, but the number of degrees of freedom associated with the system is too large, and they may be too far apart, to make recovering this information feasible. 

For thermalization to work on systems in any state, it must work on eigenstates of the many-body Hamiltonian. Since eigenstates don't evolve in time, subsystems must look thermal when the whole system is in an eigenstate, even before the long-time limit. This argument is referred to as the Eigenstate Thermalization Hypothesis (ETH)~\cite{Deutsch91, Srednicki1994, Rigol2008, Nandkishore2015}. Of course, localized systems do not obey the ETH because they do not thermalize. This thesis will study thermalizing systems.

Since information initially localized in a subsystem spreads to other degrees of freedom, the system acts as a bath for its subsystems. These degrees of freedom become entangled with the subsystem so that the state of the subsystem cannot be described without knowledge of the whole, making it look thermal. Ref.~\cite{Nandkishore2015} suggests the fundamental characteristic of a bath is its ability to entangle its degrees of freedom with those of the subsystem.

After realizing that thermalization leads to information spreading (and locally information loss), the next point of study becomes its dynamics. Scrambling in black holes can be described using holography~\cite{Sekino2008, Shenker2014}. Scrambling has been studied in conformal field theories~\cite{Calabrese2005} and spin chains with integrable and non-integrable Hamiltonian evolution~\cite{Fagotti2008, Luchli2008, Kim2013, Baradson2012}. Random unitary dynamics, discussed in Sec.~\ref{sec:circuits}, provide another setting to study spreading. This thesis will focus on entanglement dynamics in spin chains, with both time-independent Hamiltonian and random unitary dynamics. 

\hrule 

Motivate information speeds.

Scrambling, unitarity and information loss?

Why do we care about asymmetry?

Entropy can be used to study topological properties in systems with mass gaps~\cite{Kitaev2006}. In conformal field theories entropy informs the scale of the renormalization group flow~\cite{Casini2011, Myers2011}.

The second half of this thesis studies circuits... \emph{cite circuit papers here}