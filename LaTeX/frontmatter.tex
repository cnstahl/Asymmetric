{\setstretch{1.5}
\vspace*{.2in}
\begin{center}
	\huge{\textbf{Operator and Entanglement Dynamics \\
			        in Asymmetric Quantum Systems}}
\end{center}
\vspace{.6in}
\sc
\begin{center}
	\LARGE{Charles Nicholas Stahl}
\end{center}
\vspace{.6in}
\begin{center}
\today
\end{center}
\vspace{.6in}
\begin{center}
	{\Large Advised by Professor David Huse} \\
	Second Reader: Professor Shivaji Sondhi
\end{center}
\vspace{.6in}
\begin{center}
	Submitted in partial fulfillment \\
	of the Requirements for the \\
	Degree of Bachelor of Arts
\end{center}
}
\newpage

\begin{abstract}
	Thermalization is an important aspect in quantum physics from condensed matter to black holes. It allows initially local information to be spread and hidden throughout a system. This spreading happens at a finite speed, and can be quantified using the butterfly velocity $v_B$ or the entanglement velocity $v_E$. Although many sources have explored these sources, fewer explore the ratio $v_E/v_B$, and little work has been in describing asymmetric butterfly velocities. In this thesis we find both a time-independent system and a quantum circuit with different butterfly velocities in either direction, which we call $v_{B\pm}$. Although in the Hamiltonian system the two velocities are still close in magnitude, the ratio $v_{B+}/v_{B-}$ can be made arbitrarily large in the circuit model.\blfootnote{I pledge my honor that this paper represents my own work in accordance with University regulations.}
\end{abstract}

\newpage

\section*{Acknowledgments}

Thank you, Professor Huse, for making this project possible. Your availability and willingness to meet made my senior thesis process tractable, interesting, and definitely fun. From patiently explaining the necessary background material, even when I had just asked the same questions last week, to suggesting useful new questions to ask and helping to interpret the answers, you made my foray into quantum information dynamics painless and engaging. 

Professor Sondhi, thank you for your willingness to expand my physics knowledge in any way, from advising my summer and future plans, to agreeing to take on an extra math class, to being my second reader, I will always appreciate your helpfulness throughout my last two years in the department. 

Thank you Witherspoon 517+, for simultaneously backing me up and pushing me forward  since freshman year. I can't imagine being placed in a better freshman hall. I can never catch a break with you guys, but that constant needling got me here, so I can't really complain.

To the seniors and assorted juniors who have made the physics department the greatest at the University, thank you for getting me through four years of classes and independent work without losing my mind. I'm lucky to have been in such a cohesive department with so much positive encouragement between students.

Mom, Dad, Maria, thank you for being the smartest, funnest, and most interesting family I've ever had. You have instilled in me a love for science, a love for wit, and a love for knowledge that have made me who I am today. Mom and Dad, I know I would never be here without you. And Maria, I am so lucky to have been able to spend so much time with you over the past two years on campus. You're an inspiration.

And as always, thank you Haley for your wonderful five years of love and support. Knowing that, no matter how hard a day has been, you will be ready to listen and understand on the phone gives me a fabulous feeling of peace. I can't wait to see where our next five years of adventures take us. I love you to the moon and back!

\newpage

\tableofcontents