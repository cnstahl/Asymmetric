{\setstretch{1.5}
\vspace*{.2in}
\begin{center}
	\huge{\textbf{Operator and Entanglement Dynamics \\
			        in Asymmetric Quantum Systems}}
\end{center}
\vspace{.6in}
\sc
\begin{center}
	\LARGE{Charles Nicholas Stahl}
\end{center}
\vspace{.6in}
\begin{center}
\today
\end{center}
\vspace{.6in}
\begin{center}
	{\Large Advised by Professor David Huse} \\
	Second Reader: Professor Shivaji Sondhi
\end{center}
\vspace{.6in}
\begin{center}
	Submitted in partial fulfillment \\
	of the Requirements for the \\
	Degree of Bachelor of Arts
\end{center}
}
\newpage

\begin{abstract}
	Thermalization is an important aspect in quantum physics from condensed matter to black holes. It allows initially local information to be spread and hidden throughout a system. This spreading happens at a finite speed, and can be quantified using the butterfly velocity $v_B$ or the entanglement velocity $v_E$. Although many sources have explored these sources, fewer explore the ratio $v_E/v_B$, and little work has been in describing asymmetric butterfly velocities. In this thesis we find both a time-independent system and a quantum circuit with different butterfly velocities in either direction, which we call $v_{B\pm}$. Although in the Hamiltonian system the two velocities are still close in magnitude, the ratio $v_{B+}/v_{B-}$ can be made arbitrarily large in the circuit model.\blfootnote{I pledge my honor that this paper represents my own work in accordance with University regulations.}
\end{abstract}

\newpage

\section*{Acknowledgments}

\newpage

\tableofcontents