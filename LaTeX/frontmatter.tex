{\setstretch{1.5}
\vspace*{.2in}
\begin{center}
	\huge{\textbf{Operator and Entanglement Dynamics \\
			        in Asymmetric Quantum Systems}}
\end{center}
\vspace{.6in}
\sc
\begin{center}
	\LARGE{Charles Nicholas Stahl}
\end{center}
\vspace{.6in}
\begin{center}
\today
\end{center}
\vspace{.6in}
\begin{center}
	{\Large Advised by Professor David Huse} \\
	Second Reader: Professor Shivaji Sondhi
\end{center}
\vspace{.6in}
\begin{center}
	Submitted in partial fulfillment \\
	of the Requirements for the \\
	Degree of Bachelor of Arts
\end{center}
}
\newpage

\begin{abstract}
	Thermalization is an important aspect in quantum physics from condensed matter to black holes. It allows initially local information to be spread and hidden throughout a system. This spreading happens at a finite speed, and can be quantified using the butterfly velocity $v_B$ or the entanglement velocity $v_E$. These speeds are well-studied, and are independent of each other up to the constraint $v_B>v_E$. Although it is possible to have a direction-dependent $v_B$, little work has been done to study systems like this. In this thesis we study two systems on spin chains with asymmetric butterfly velocities, which we call $v_{B\pm}$. In the first, a system with a time-independent Hamiltonian, we study $v_B$ through operator spreading. We show that the system is slightly asymmetric, with $v_{B+}>v_{B-}$. The second system is a quantum circuit with random unitary dynamics. Using entanglement dynamics to measure the butterfly velocity, we show that these systems can have $v_{B+}/v_{B-}$ arbitrarily large.\blfootnote{I pledge my honor that this paper represents my own work in accordance with University regulations. \vspace*{1in}}
\end{abstract}

\newpage

\section*{Acknowledgments}

Thank you, Professor Huse, for making this project possible. Your availability and willingness to meet made my senior thesis process tractable, interesting, and so much fun. From patiently explaining the necessary background material, even when I had just asked the same questions last week, to suggesting useful new directions to explore and helping me interpret the results, you made my foray into quantum information dynamics painless and engaging. 

Professor Sondhi, thank you for all you have done to expand my physics knowledge, from advising my summer and future plans, to agreeing to advise an extra math class this semester, to being my second reader, I will always appreciate your helpfulness throughout my last two years in the department. 

Thank you Witherspoon 517+, for simultaneously backing me up and pushing me forward  since freshman year. I can't imagine being placed in a better freshman hall. I can never catch a break with you guys, but that constant needling got me here, so I can't really complain.

To the seniors and assorted juniors who have made the physics department a comfortable place for me, thank you for getting me through four years of classes and independent work without losing my mind. I'm lucky to have been in such a cohesive department with so much positive encouragement between students.

Mom, Dad, Maria, thank you for being the smartest, funnest, and most interesting family I've ever had. You have instilled in me a love for science, a love for wit, and a love for knowledge that have made me who I am today. Mom and Dad, I know I would never be here without you. And Maria, I am so lucky to have been able to spend so much time with you over the past two years on campus. You're an inspiration.

And as always, thank you Haley for five wonderful years of love and support. Knowing that, no matter how hard a day has been, you will be ready to listen and understand on the phone gives me a fabulous feeling of peace. I can't wait to see where our next five years of adventures take us. I love you to the moon and back!

\newpage

\tableofcontents