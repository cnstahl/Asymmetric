\section{Operator Spreading in Time-Independent Hamiltonian Systems} \label{sec:opsp}

Just do all intro here, and then decide later what to put into the circuit section

Thermalization 
  vs localization 
  happens by spreading information
  mention that this is how information loss happens

Discuss information movement in Time-Independent Hamiltonian systems
  Find a source for this
  Ideally introduce OTOC and Pauli end weight here
  Also entanglement possibly
  What is the relationship between entropy and OTOC?
  Argument about extremal slope
    From Nahum: stairs to separate $v_B$ and $v_E$ - It is possible to make $v_E << v_B$ in quantum circuit architectures [CITE], which will be discussed in section~\ref{sec:circuits}
    

For now use Jonay paper, but hopefully find another. Could use Keyserlingk but it would be nice to save that for the discussion of random unitary dynamics.

Although most sources consider symmetric dynamics [CITE] this is not a requirement. Find the Nahum paper that discusses asymmetric systems.

\hrule

All systems considered in this thesis will exist on spin chains, one-dimensional collection of quantum degrees of freedom. Initially we consider systems with $q=2$ degrees of freedom at each site, such as a chain of spin-$\half$ particles. Later, we will consider sites with more degrees of freedom. 

Under a time-independent Hamiltonian $H$, states of the system evolve in the Schr\"odinger picture as 
\begin{align}
\ket{\psi(t)} = U(t)\ket{\psi(0)},\quad U(t) = \e^{-iHt}. \label{eqn:shro}
\end{align}
This evolution can be generalized to a time-dependent Hamiltonian by evolving with the time-ordered unitary operator
\begin{align}
U(t,t_0) = \text{T}\e^{-i\int_{t_0}^td\tau H(\tau)}.\nonumber
\end{align}
As we will consider time-independent Hamiltonians, we only need time evolution operators of the form of equation~\ref{eqn:shro}.

Instead of evolving states, it is possible to evolve operators under the Heisenberg picture. In order to preserve the time dependence of expectation values, the operators must evolve as 
\begin{align}
A(t) = U^\dag(t)\,A(0)\,U(t) = \e^{iHt}A(0)\,\e^{-iHt}.\label{eqn:heis}
\end{align}

One remark worth making about time evolution in different pictures is that operators aren't the only the only important Hermitian matrix in the system. The density matrix $\rho$ describes the state of the system, and in pure states is $\rho = \ket{\psi}\bra{\psi}$. Density matrices are more general than kets, though, because they can represent mixed states. In the Schr\"odinger picture $\rho$ evolves the way its construction would imply
\begin{align}
\rho(t) = \e^{-iHt}\rho(0)\e^{iHt}
\end{align}
while it does not evolve in the Heisenberg picture. This is the opposite of observables, so when discussing the evolution of matrices it is necessary to specify whether they are observables or density matrices.

\subsection{Pauli Strings and Pauli Weight} \label{sub:pauli}

This subsection is based on~\cite{Keyserlingk}. When discussing operator spreading it is convenient to decompose operators that may act on very high dimensional Hilbert spaces into the Pauli basis. The basis operators are tensor products of Pauli matrices. Eventually we will decompose operators that are initially local, but any operator can be decomposed in this manner.

For single sites with Hilbert spaces of complex dimension $q$, the space of Hermitian operators is $q^2$-dimensional.\footnote{Check this. Something is off.} For $q=2$ the basis operators are $X, Y, Z, I$. In general, the operators will be of the form 
\begin{align}
\sigma^\mu = X^{\mu_1}Z^{\mu_2},
\end{align}
where $\mu_1, \mu_2\in\{0,1,\dots,q-1\}$\footnote{How to get Y?}. Under the matrix norm $||M|| = \text{tr}(M^\dag M)/q$, this basis is orthonormal:
\begin{align}
\th{q}\text{tr}(\sigma^{\mu\dag}\sigma^\nu) &= \th{q}\text{tr}(Z^{\mu_2\dag}
	X^{\mu_1\dag}X^{\nu_1}Z^{\nu_2})\nn
&= \delta_{\mu\nu}.\label{eqn:orthonorm}
\end{align}

A general operator $A$ evolves into
\begin{align}
A(t) = U^\dag(t)A\,U(t) = \sum_\nu c_\nu(t)\sigma^\nu.\label{eqn:decomp}
\end{align}
Due to the orthonormality, the coefficients are 
\begin{align}
c_\nu(t) = \th{q}\text{tr}(\sigma^{\nu\dag}A(t))
\end{align}
and obey $\sum_\nu \abs{c_\nu}^2 = 1$.

With $c_\nu(t)$ in hand, we can define the Pauli weight $W(i,t)$ as how much of the weight of the operator is on Pauli strings that end on site $i$:
\begin{align}
W(i,t) = \sum_\nu\abs{c_\nu(t)}^2\delta(\text{end}(\nu)=i).
	\label{eqn:endweight}
\end{align}
The delta function constrains the sum to be only over $\nu$ such that $\sigma^\nu$ has a non-identity at site $i$ and identities at all sites right of $i$. Reference~\cite{Keyserlingk} refers to this quantity as $\rho$ to emphasize its hydrodynamic evolution. It is possible to define an analogous quantity with the sum over strings that begin on site $i$- those which have non-identities on all sites left of $i$. In that case the quantity in equation~\ref{eqn:endweight} can be called $W_R(i,t)$ while the weight of sites that start on $i$ is $W_L(i,t)$.

As an example, write
\begin{align}
A = X_1\otimes Y_2\otimes Y_3\otimes I_4 + I_1\otimes I_2\otimes Z_3\otimes Z_4,
	\nonumber
\end{align}
where the subscript designates the site on which each operator acts. This can be shortened to
\begin{align}
A = XYYI + IIZZ\label{eqn:inioper}.
\end{align}

This decomposition is particularly useful when the initial operator is local. This means that all strings in the Pauli decomposition contain non-identiy operators at a single site. Then the end weights describe how far the operator has spread throughout the system due to the unitary dynamics. \emph{Mention thermalization} \emph{figure from Jonay}

The Pauli weight is closely related to the Out-of-Time-Ordered Correlator (OTOC). Consider an initial operator, say, $A(0)=Z_1=ZII\dots$. This will commute with operators that are local at other sites and later times, $B(x,t)$. However, in general the time-evolved $A(t)$ will include Pauli strings that have non-identity operators at all sites. This can be seen through the Baker-Campbell-Hausdorff expansion~\cite{Roberts2016} of the time evolution
\begin{align}
A(t) &= \e^{iHt}A(0)\e^{-iHt} \nn
&= \sum_k\frac{(it)^k}{k!}[H,[H,\dots[H,A(0)]\dots]]
\end{align}

Find Pauli weight. Discuss name, way to calculate. Relate to OTOC

Find weight at site.

\subsection{Entanglement Entropy} \label{sub:intro}

Quantum entanglement describes aspects of branches of physics from high energy and quantum information theory to experimental studies of cold atomic gases. Although entanglement is so widely studied, its dynamics are less well understood. The dynamics of the entanglement entropy are closely related to the speed at which information travels or spreads. 
%One way to study this topic is to consider entanglement dynamics of spin chains. 

Entanglement entropy provides one way to quantify the entanglement and is defined as follows. For a system $AB$ divisible into subsystems $A$ and $B$, the reduced density matrices $\rho_A$ and $\rho_B$ are the full density matrix $\rho_{AB}$ traced over subsystem $B$ and $A$, respectively. If the full system is in a pure state, the $n$th Renyi entropy of density matrix $\rho$ is 
\begin{align}
S_n = \th{1-n}\log\left(\Tr\rho^n\right). \label{eqn:renyi}
\end{align}
In the limit $n\to1$ this becomes the von Neumann entropy
\begin{align}
S_{vN} = -\Tr\rho\log\rho,
\end{align}
the analogue of the classical Shannon entropy. These entropies are maximized by maximally mixed states, with entropy $N\log q$ for $N$-site systems with $q$-dimensional Hilbert spaces at each site.

Introduce entropy bounds here.

References~\cite{Keyserlingk, Jonay} discuss the speed of entanglement in brickwork models using related concepts called out of time order commutator (OTOC) and operator density. Reference~\cite{Zhou2017} quantifies the scrambling using the operator entanglement entropy opEE of the time evolution operator.

\subsubsection{Entropy Constraints} \label{subsub:constraints}

The following description is largely taken from~\cite{Nahum2017}. Consider a spin chain of N sites with dimension $q$. $q=2$ corresponds to spin-$\half$ particles, $q=3$ corresponds to spin-1 particles, etc. Sites are labeled by $i=1,\dots N$, while the bonds between sites are labeled by $x = 1,\dots N-1$. After cutting the system at bond $x$, define the entropy across this cut as the bipartite entanglement entropy of all sites to the right of $x$. If the whole chain is in a pure state, this is equal to the bipartite entanglement entropy of all sites to the left of $x$.

The von Neumann entropy at cut $x$ is
\begin{align}
S(x) = \-\Tr\rho_x\log\rho_x, \label{eqn:vonneu}
\end{align}
where $\rho_x$ is the density matrix of the system with all sites left of $x$ traced out, and for convenience logarithms are taken base $q$. Classically, for an arbitrary system decomposable into subsystems $A$ and $B$, the entropies satisfy $\max(S(A), S(B)) \leq S(AB)\leq S(A) + S(B)$. In quantum mechanics, this is replaced by the subadditivity of the von Neumann entropy 
\begin{align}
\left|S(A)-S(B)\right| \leq S(AB)\leq S(A) + S(B). \label{eqn:subadd}
\end{align}
If we take subsystem $A$ to be the single site between cuts $x$ and $x+1$ and subsystem $B$ to be all sites right of $x+1$, this becomes
\begin{align}
\left|S_1 - S(x+1)\right| \leq S(x) \leq S_1 + S(x+1),
\end{align}
where $S_1$ denotes the entropy of the single site between cuts $x$ and $x+1$. After some rearranging this can be written $\left|S(x+1) - S(x)\right| \leq S_1$. However, since the single site is $q$ dimensional, $S_1 \leq \log q = 1$, explaining the use of $q$ for the base. The preceding arguments taken together give the constraint
\begin{align}
\left|S(x+1) - S(x)\right| \leq 1. \label{eqn:offbyone}
\end{align}
