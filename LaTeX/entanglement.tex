\documentclass[11pt]{article}

\usepackage[hyperfootnotes=false]{hyperref}
\usepackage[margin=1in]{geometry}                                              
\usepackage{amsmath,amsthm,amssymb}      
\usepackage{titlesec}                                                         
\usepackage{bm}        
\usepackage{cprotect}                                
\usepackage{savetrees} 
\usepackage{bbold}
\usepackage{abstract}
                                          
\usepackage{graphicx}                                      
\graphicspath{{../figures/}}

\usepackage{biblatex}
\bibliography{bib.bib}

\titleformat{\subsection}[runin]
{\normalfont\large\bfseries}{\thesubsection}{1em}{}

\renewcommand{\bf}{\mathbf}
\renewcommand{\cal}{\mathcal}
\newcommand{\pd}[2]{\frac{\partial #1}{\partial #2}}
\newcommand{\pdn}[3]{\frac{\partial^{#3} #1}{\partial #2^{#3}}}
\newcommand{\pdop}[1]{\frac{\partial}{\partial #1}}
\newcommand{\nd}[2]{\frac{d #1}{d #2}}
\newcommand{\ndn}[3]{\frac{d^{#3} #1}{d #2^{#3}}}
\newcommand{\ndop}[1]{\frac{d}{d #1}}
\newcommand{\dt}{\frac{d}{dt}}
\newcommand{\grad}{\bm\nabla}
\newcommand{\cross}{\times}
\newcommand{\curl}{\grad\cross}
\newcommand{\imp}{\Longrightarrow\quad}
\newcommand{\abs}[1]{\left|#1\right|}
\newcommand{\half}{\frac{1}{2}}
\newcommand{\third}{\frac{1}{3}}
\renewcommand{\th}[1]{\frac{1}{#1}}
\renewcommand{\k}{4\pi\epsilon_0}
\newcommand{\eps}{\epsilon_0}
\newcommand{\intt}{\int_{t_1}^{t_2}}
\newcommand{\inti}{\int_{-\infty}^{+\infty}}
\newcommand{\ex}[1]{\left\langle #1 \right\rangle}
\newcommand{\oom}[1]{\times 10^{#1}}
\renewcommand{\d}{\delta}
\newcommand{\e}{\text{e}}
\renewcommand{\l}{\ell}
\newcommand{\om}{\omega}
\newcommand{\h}{\hbar}
\newcommand{\ket}[1]{\left|#1\right\rangle}
\newcommand{\bra}[1]{\left\langle#1\right|}
\newcommand{\braket}[2]{\left\langle#1\middle|#2\right\rangle}
\newcommand{\brakett}[3]{\left\langle#1\middle|#2\middle|#3\right\rangle}
\newcommand{\nn}{\nonumber\\}

\renewcommand{\thesection}{\arabic{section}:}
\renewcommand{\thesubsection}{(\alph{subsection})}

\begin{document}

\title{Entanglement Dynamics in Random Unitary Circuits}
\author{Charles Stahl}

\maketitle

\section{Ergodicity and Stability} \label{sec:erg}

A Markov process is one in which the future state depends only on the current state, not the past. Label the states $s_i$ and define $p_{i,t}$ as the probability that the system is in state $s_i$ at time $t$. For Markov processes it is possible to define the transition matrix... Since the product of the transition matrix and a probability vector gives the probabilities at the next time step, the transition matrix for $T$ time steps is just $P^T$. Under certain conditions\footnote{Figure these out.} the multi-step transition matrix approaches a constant matrix with all columns equal to the same vector $v^*$,
\begin{align}
\lim\limits_{T\to \infty}P^T = P^* = \begin{bmatrix}
\vdots & \vdots &  & \vdots\\
v^* & v^* & \cdots & v^*\\
\vdots & \vdots &  & \vdots\\
\end{bmatrix}.
\end{align}
Then the probability after a long time is the vector $v^*$ for any initial state.

For the 1-stair circuit, the transition matrix is the matrix product of single-site transition matrices $P_{N} = \prod_NP_{1}\otimes$, where the single-state transition matrix is
\begin{align}
P_1 = \begin{bmatrix}
1-\frac{1+m}{2}\Gamma & \frac{1+m}{2}\Gamma & 0 & 0\\
0 & 1-\Gamma & \Gamma & 0\\
\frac{1-m}{2}\Gamma & 0 & 1-\Gamma & \frac{1+m}{2}\Gamma\\
0 & \frac{1-m}{2}\Gamma & 0 & 1 - \frac{1-m}{2}\Gamma
\end{bmatrix}.
\end{align}
The equilibrium state is
\begin{align}
v^* = \begin{pmatrix}
content...
\end{pmatrix}
\end{align}

\end{document}
